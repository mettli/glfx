%\documentclass[11pt,a4paper,draft,titlepage]{article}
\documentclass[11pt,a4paper,final,titlepage]{article}

\usepackage{fixltx2e}
\usepackage{fullpage}
\usepackage[svgnames]{xcolor}
\usepackage{amsmath}
\usepackage{amsfonts}
\usepackage{amsthm}
\usepackage{graphicx,float,wrapfig,tabulary}
\usepackage[justification=Centering,singlelinecheck=false,labelformat=empty]{subfig}
\usepackage{showkeys}
\usepackage{ulem}
\usepackage{lipsum}
\usepackage{fixme}
\usepackage{aliascnt}
\usepackage{wasysym,marvosym,textcomp}
\usepackage{fontspec,xunicode,xltxtra}
\usepackage[naturalnames,colorlinks=true,linkcolor=blue,xetex]{hyperref}
\usepackage[ruled,vlined,linesnumbered,algonl]{algorithm2e}
\usepackage[final]{listings}
\usepackage{attachfile}

\parskip .3cm

\fxsetup{
    status=draft,
    author={TODO},
    layout=inline, % also try inline, footnote or pdfnote
    theme=color
}

% Fonts
\setmainfont[Mapping=tex-text]{Cambria}
\setsansfont{Microsoft Sans Serif}
\setmonofont{Consolas}

% Define C++ syntax for listings
\lstset{language=C++,
%caption=\lstname,
%backgroundcolor=\color{LightGrey},
tabsize=4,
rulecolor=,
%basicstyle=\scriptsize,
%basicstyle=\small,
basicstyle=\normalsize\ttfamily,
upquote=true,
aboveskip={1.5\baselineskip},
columns=fixed,
showstringspaces=false,
extendedchars=true,
breaklines=true,
%prebreak = \raisebox{-1ex}[0ex][0ex]{\ensuremath{\hookleftarrow}},
%frame=single,
showtabs=false,
showspaces=false,
showstringspaces=false,
identifierstyle=\bfseries,
keywordstyle=\color[rgb]{0,0,.5},
%keywordstyle=\bfseries\ttfamily\color[rgb]{0,0,1},
commentstyle=\color[rgb]{0.133,0.545,0.133},
stringstyle=\color[rgb]{0.627,0.126,0.941},
xleftmargin=0.3cm,
% Add some keywords for GLSL
morekeywords={vec4,vec3,vec2,in,out,flat,sampler2DArray,texture,gl_Position,layout,shader,interface,
uniform,mat4,program,vs,fs,stages,sampler},
}

\normalem

% Theorems and such
\newtheorem{theorem}{Theorem}
\newaliascnt{lemma}{theorem}
\newtheorem{lemma}[lemma]{Lemma}
\aliascntresetthe{lemma}
% In case I want unique numbering for lemmas then comment
% three previous lines and uncomment next line
%\newtheorem{lemma}{Lemma}

% hyperref names redo (and additions)
\let\orgautoref\autoref
\renewcommand{\autoref}{%
\def\sectionautorefname{Sec.}%
\def\subsectionautorefname{Sec.}%
\def\algocflineautorefname{Alg.}%
\def\lemmaautorefname{Lemma}%
\orgautoref}
\providecommand{\Autoref}{%
\def\sectionautorefname{Section}%
\def\subsectionautorefname{Section}%
\def\algocflineautorefname{Algorithm}%
\def\lemmaautorefname{Lemma}%
\orgautoref}

% Algorithms stuff
\SetKwInOut{Input}{input}
\SetKwInOut{Output}{output}
%\dontprintsemicolon

\newcommand{\LL}{\mathcal{L}}
\newcommand{\FF}{\mathcal{F}}
\newcommand{\EE}{\mathbb{E}}
\newcommand{\rs}{\mathbb{R}}
\newcommand{\lp}{\left(}
\newcommand{\rp}{\right)}
\newcommand{\slp}{\left[}
\newcommand{\srp}{\right]}
\newcommand{\lnm}{\left\|}
\newcommand{\rnm}{\right\|}
\newcommand{\ld}{\left<}
\newcommand{\rd}{\right>}
\newcommand{\lb}{\left\{}
\newcommand{\rb}{\right\}}
\newcommand{\partiald}[2]{\frac{\partial#1}{\partial#2}}
\newcommand{\fulld}[3][]{\frac{d^{#1}}{d{#3}^{#1}}\lp#2\rp}
\newcommand{\fullds}[3][]{\frac{d^{#1}#2}{d{#3}^{#1}}}
\newcommand{\ft}[1]{\mathcal{F}\lb #1\rb}
\newcommand{\ift}[1]{\mathcal{F}^{-1}\lb #1\rb}
\newcommand{\sfrac}[2]{{}^{#1}/_{#2}\;}
\newcommand{\sinc}{\operatorname{sinc}}
\newcommand{\sgn}{\operatorname{sgn}}
\newcommand{\diag}{\operatorname{diag}}
\newcommand{\tr}{\operatorname{tr}}

\begin{document}
\title{GLFX 0.6}
\author{Max Eisenstein\\Etay Meiri}
\date{\today}
\maketitle

\listoffixmes

\tableofcontents

\pagebreak

\section{Introduction}\label{sec:intro}
This document describes the GLFX parser: what it is, what its relation to GLSL,
what syntax it has, how to use it and its shortcomings. We assume that the reader is familiar with the
4.3 OpenGL and GLSL specifications (from now on, the OpenGL spec).

\subsection{What is GLFX?}
In simplest terms, GLFX is a source to source parser which is managed by a set of interface functions
(from now on, in short, API).
It exists in order to fill a few gaps which were formed in the OpenGL spec:
\begin{itemize}
\item The lack of any reference to the underlying file system in the context of shader programs.
As far as the OpenGL spec cares you can hold your shader code in static strings. Basic file system access
is a fairly standardized process today, so there's no reason to be so harsh as OpenGL.
\item The GLSL compiler expects a single entry point in the shader source, namely the
{\texttt{\textbf{main}}} function. This makes perfect sense for C/C++ programs, and since
GLSL syntax stems from C it inherited the same entry point. However, the OpenGL
pipeline consists of stages, while each programmable stage has its own code.
Hence, using the same name for all entry points from different stages just doesn't make sense.
\item The lack of a simple and intuitive header mechanism (combined with the previous bullet) completely
shuts down all simple ways to share code and definitions between programmable stages
(either from the same program or different programs).
Also, the \texttt{ARB\_\-shading\_\-language\_\-include} extension doesn't make
it any easier to share definitions between your C++ code and your shader code, basically dropping
you to square one if that was your intent.
Besides that, the extension isn't very intuitive to use. And finally, it's just an extension, not core.
\item Except for \texttt{\textbf{Create\-Shader\-Program}}, there are no functions that act as
wrappers for lower level API.
This forces the users to perform the exact same calls every time they want to construct a program.
\item In many cases it makes more sense define a sampler from within the shader, since the shader
should know how to sample from the resource.
\end{itemize}

\subsection{Relation to GLSL}
GLFX tries to fill the previously mentioned gaps in a manner which will be most consistent with
the spirit of GLSL and the pipelined nature of OpenGL. Since, at this point, it's a relatively simple
package, some shortcuts were made in order to deliver usability while sacrificing consistency.
These shortcomings will be addressed in future releases.

If you're familiar with GLSL, using GLFX should be a straightforward procedure. If you're familiar
with HLSL, you'll see similar ideas to what was used in Microsoft's effects package.

\pagebreak
\section{The GLFX API}

\subsection{Manipulating effect objects}
Effect objects are associated with effect files (which will be discussed in \autoref{sec:effFiles})
which hold all the information required to compile and link programs.

In order to generate an effect object, call:

\begin{lstlisting}
int glfxGenEffect();
\end{lstlisting}
The return value is the identifier of the effect object.

In order to delete effect object, call:

\begin{lstlisting}
void glfxDeleteEffect(int effect);
\end{lstlisting}
With \texttt{\textbf{effect}} as the identifier of the effect.

In order to attach an effect file and to parse it, call:

\begin{lstlisting}
bool glfxParseEffectFromFile(int effect, const char* file);
\end{lstlisting}
Which returns whether parsing was successful. \texttt{\textbf{effect}} is the identifier.
\texttt{\textbf{file}} is a null terminated string of the file to parse, while the path is relative
to the current path. If parsing was unsuccessful, errors are logged and can be extracted later.

In order to compile a named program from the effect file, use:

\begin{lstlisting}
int glfxCompileProgram(int effect, const char* program);
\end{lstlisting}
Supply \texttt{\textbf{effect}} as the identifier, and \texttt{\textbf{program}} as the null
terminated string containing the name of the program.
In case of successful compilation an OpenGL program object is returned. If compilation has failed
then $-1$ is returned, and errors are logged. Program syntax is discussed in \autoref{sec:programs_def}.
Once the program is returned, all involved shaders are detached and deleted. This means that relinking
programs outside of GLFX is impossible.

In order to generate a named sampler, call:

\begin{lstlisting}
int glfxGenerateSampler(int effect, const char* sampler);
\end{lstlisting}
With \texttt{\textbf{effect}} as the identifier and \texttt{\textbf{sampler}} as the null terminated
string containing the type or the name of the sampler.
In case of error $-1$ is returned, otherwise an OpenGL sampler object is created and returned.
Sampler syntax is discussed in \autoref{sec:samplers}.

\subsection{Handling errors}
The log is accumulated until retrieved by any of the APIs.

A complete error log (since the last retrieval) can be obtained by calling:
\begin{lstlisting}
void glfxGetEffectLog(int effect, char* log, int bufSize);
\end{lstlisting}
Where \texttt{\textbf{effect}} is the identifier, \texttt{\textbf{log}} is the destination
string and \texttt{\textbf{bufSize}} is the maximum size of the string.

Alternatively, if C++ is used, a simpler API call is available:
\begin{lstlisting}
std::string glfxGetEffectLog(int effect);
\end{lstlisting}
Which returns a \texttt{\textbf{std::string}} object.

Currently there are three types of errors:
\begin{enumerate}
\item General parsing errors
\item GLSL errors
\item Sampler generation errors
\end{enumerate}

\subsubsection{General parsing errors}
These errors are produced by GLFX. They occur during parsing of the effect file, usually when
a syntactic or a semantic error is encountered in GLFX specific code.
For the sake of simplicity GLFX avoids parsing things outside of its scope.
This means that GLFX cannot detect errors in GLSL code during initial parsing.

\emph{Note: GLFX's error reporting mechanism is currently limited. If an error was encountered
during parsing, the log won't be very informative. The only information that's currently supplied
is the type of the error. However, the reported line number is correct.}

\subsubsection{GLSL errors}
These errors (and warnings) are generated during compilation (and linkage). Their quality depends on
the GLSL compiler. The line numbers are in the effect file when $(0)$ is prefixed. When $(i)$
is prefixed then the error has occurred in the $i$'th header file.

\subsubsection{Sampler generation errors}
These errors occur when sampler generation was failed. This happens usually due to incorrect
parameter in the effect file (in the sampler definition).

\pagebreak
\section{Effect files}\label{sec:effFiles}
Effect files are essentially a collection of shaders, helper functions, various structures, header
includes, sampler definitions and program definitions.

\subsection{Shader entry point}
In contrast to plain GLSL, where the entry point is the rigid \texttt{\textbf{main}}, an effect
file can define multiple entry points. Each such entry point is a function of the type
\emph{shader}, with a parameter list and optional register definitions. For example:
\begin{lstlisting}
shader MyVertexShader(in vec4 lPos:0)
{
	gl_Position=mPVW*lPos;
}
\end{lstlisting}
Here we define a vertex shader with \texttt{\textbf{lPos}} as the vertex input position, which is located
in register $0$. As per GLSL spec, supplying register numbers is completely optional. However, it's
strongly advised to do so when linking separable programs (discussed in \autoref{sec:programs_def}).

\subsection{Interfaces}
The interface mechanism connects different programmable stages. While not
strictly required to connect stages, it's the advised way to do so since
it gives the programmer more control than structures. This is especially important for separable
programs.

Here is an example of how to use interfaces:
\begin{lstlisting}
sampler2DArray texArray;
uniform mat4 mPVW;

interface VStoFS {
	layout(location=0) vec2 texC;
	layout(location=1) flat int sampleTex;
};

shader MyVS(in vec4 lPos:0, in vec2 texC:1, in int sampleTex:2, out VStoFS outputs)
{
	gl_Position=mPVW*lPos;
	outputs.texC=texC;
	outputs.sampleTex=sampleTex;
}

shader MyFS(in VStoFS inputs, out vec4 fCol:0)
{
	vec3 texC(inputs.texC, inputs.sampleTex);
	fCol=texture(texArray, texC);
}
\end{lstlisting}

There's a secondary value to using the interface mechanism. Currently GLFX still doesn't
support all qualifiers if passed in the parameter line, but inside interface there are no restrictions
since it's not parsed by GLFX, but rather passed directly to GLSL compiler. Though this
will be fixed in future releases. Also, interface blocks will be parsed as well in order to
simplify layout specification.

\subsection{Program definition}\label{sec:programs_def}
In order to combine all stages into a single program a definition should be made for the
\texttt{\textbf{glfx\-Compile\-Program}} API to recognize the program. This is done by declaring
what shader corresponds to what stage. In case of non-separable programs
the \texttt{\textbf{program}} keyword should be used as follows:

\begin{lstlisting}
program MyProgram {
	vs(430)=MyVS();
	fs(430)=MyFS();
};
\end{lstlisting}

The possible stages are: \texttt{\textbf{vs,hs,ds,gs,fs,cs}}; for Vertex, Hull, Domain, Geometry, Fragment
and Compute shaders, respectively. The numbering in the parenthesis specifies the GLSL
version to compile against.

In order to link separable program use the keyword \texttt{\textbf{stages}} instead of
\texttt{\textbf{program}}.

In order to specify layout qualifiers, use \texttt{\textbf{in}} and \texttt{\textbf{out}} keywords
as follows:
\begin{lstlisting}
stages MyGeometry {
	gs(430)=MyGS()
	: in(triangles_adjacency, invocations=6)
	, out(triangle_strip, max_vertices=18, stream=1);
};
\end{lstlisting}

\pagebreak
\section{Headers}
GLFX has a preliminary support for header files with. The usage is just like in C++ source
code:
\begin{lstlisting}
#include "myheader.h"
\end{lstlisting}

GLFX doesn't actually parses the header files, it just glues them to the source. Hence nested
includes aren't supported.

\subsection{Sharing code with C++}
In order to share code between shaders and C++, predefined constants can be used as follows:
\begin{lstlisting}
#ifndef _MY_SHARED_HEADER_H
#define _MY_SHARED_HEADER_H

// shared code here

#ifndef __TIMESTAMP__

// GLSL code here

// __TIMESTAMP__
#endif
// _MY_SHARED_HEADER_H
#endif
\end{lstlisting}

\texttt{\textbf{\_\_TIMESTAMP\_\_}} is a good example of a compiler constant which isn't defined by GLSL.
It's safe to use as long as OpenGL won't specify file input.
This mechanism is convenient for shared constants, and especially for input register numbers.

\subsection{Errors in headers}
If an error is occurred in the header file during GLSL compilation, the line number
will be the line number in the header, prefixed (or postfixed) by $(i)$ for the $i$'th
included header.

\pagebreak
\section{Samplers}\label{sec:samplers}

A sampler is defined by the following syntax:
\begin{lstlisting}
sampler MySamplerName {
	state1 = value1;
	state2 = value2;
	// ...
} MyTexture1:0, MyTexture2:1;
\end{lstlisting}
Sampler's name is used by the API to retrieve the sampler. Texture names are used in the shader
as regular texture units with the specified binding points.
State values determine the sampler and the texture units.

\subsection{Images}
In order to use images, just replace the \texttt{\textbf{sampler}} keyword with \texttt{\textbf{image}}
keyword.

\subsection{Selecting the texture type}
The type of the texture can be anything that is defined in the GLSL spec and is specified with
the following state values:
\begin{table}[h]
\centering
\begin{tabular}{|c|c|c|c|}
\hline
Description & State keyword & State values & Default value\\
\hline\hline
Dimension & \texttt{\textbf{Dim}} & 1,2,3, Cube, Buffer & 2\\
\hline
Is array & \texttt{\textbf{Array}} & 0,1 & 0\\
\hline
Is multisampled & \texttt{\textbf{MS}} & 0,1 & 0\\
\hline
Is rectangle & \texttt{\textbf{Rect}} & 0,1 & 0\\
\hline
Format & \texttt{\textbf{Format}} & float, int, uint & float\\
\hline
\end{tabular}
\caption{Texture types}
\label{tab:texTypes}
\end{table}

\subsection{Selecting filtering modes}
The following filtering modes are available:
\begin{table}[h]
\centering
\begin{tabulary}{\linewidth}{|C|C|C|C|}
\hline
Description & State keyword & State values & Default value\\
\hline\hline
Minifying filter & \texttt{\textbf{MinFilter}} & point, linear, nearest, bilinear, trilinear & bilinear\\
\hline
Magnifying filter & \texttt{\textbf{MagFilter}} & point, nearest, linear, bilinear & linear\\
\hline
Anisotropic filter & \texttt{\textbf{Aniso}} & Anisotropy integer value & 1\\
\hline
\end{tabulary}
\caption{Filtering modes}
\label{tab:filtModes}
\end{table}

There's some redundancy in this table, since magnifying filters can be only linear or nearest,
but in order to supply more consistency with minifying filters, point and bilinear were inserted
as well. They still correspond to nearest and linear, respectively.

The minifying filter state value \texttt{\textbf{point}} corresponds to \texttt{\textbf{GL\_\-NEAREST}},
while \texttt{\textbf{nearest}} corresponds to \texttt{\textbf{GL\_\-NEAREST\_\-MIPMAP\_\-NEAREST}}.

\subsection{Coordinates state}
The following coordinates state modes are available:
\begin{table}[h]
\centering
\begin{tabulary}{\linewidth}{|C|C|C|C|}
\hline
Description & State keyword & State values & Default value\\
\hline\hline
Wrapping & \texttt{\textbf{WrapS, WrapT, WrapR}} &
repeat, mirroredRepeat, clampToEdge, clampToBorder & repeat\\
\hline
LOD margins & \texttt{\textbf{MinLod, MaxLod}} & $\in\slp-1000,1000\srp$ & $-1000$, $1000$\\
\hline
LOD bias & \texttt{\textbf{LodBias}} & Restricted to GL\_MAX\_TEXTURE\_LOD\_BIAS & 0\\
\hline
\end{tabulary}
\caption{Coordinates states}
\label{tab:coordModes}
\end{table}

\subsection{Comparison sampling}
When sampling from depth textures it's possible to let the hardware do the comparison.
The corresponding state values are:
\begin{table}[h]
\centering
\begin{tabulary}{\linewidth}{|C|C|C|C|}
\hline
Description & State keyword & State values & Default value\\
\hline\hline
Mode & \texttt{\textbf{CmpMode}} & none, ref & none\\
\hline
Compare function & \texttt{\textbf{CmpFunc}} &
lequal, gequal, less, greater, equal, notequal, always, never & lequal\\
\hline
\end{tabulary}
\caption{Comparison sampling state}
\label{tab:compModes}
\end{table}


\end{document}













